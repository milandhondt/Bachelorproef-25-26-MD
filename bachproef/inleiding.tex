%%=============================================================================
%% Inleiding
%%=============================================================================

\chapter{\IfLanguageName{dutch}{Inleiding}{Introduction}}%
\label{ch:inleiding}

De inleiding moet de lezer net genoeg informatie verschaffen om het onderwerp te begrijpen en in te zien waarom de onderzoeksvraag de moeite waard is om te onderzoeken. In de inleiding ga je literatuurverwijzingen beperken, zodat de tekst vlot leesbaar blijft. Je kan de inleiding verder onderverdelen in secties als dit de tekst verduidelijkt. Zaken die aan bod kunnen komen in de inleiding:

\begin{itemize}
  \item context, achtergrond
  \item afbakenen van het onderwerp
  \item verantwoording van het onderwerp, methodologie
  \item probleemstelling
  \item onderzoeksdoelstelling
  \item onderzoeksvraag
  \item \ldots
\end{itemize}

\section{\IfLanguageName{dutch}{Probleemstelling}{Problem Statement}}%
\label{sec:probleemstelling}

Binnen de opleidingsonderdelen Front-End Web Development en Web Services aan HOGENT worden studentenprojecten geëvalueerd door lectoren op basis van vooraf gedefinieerde criteria, zoals codekwaliteit, structuur, documentatie en foutafhandeling. Deze evaluaties gebeuren momenteel volledig handmatig, wat voor de lectoren een enorm tijdsintensief en repetitief proces is.

\vspace{1em}

Verder gaat er een aanzienlijk deel van de evaluatietijd per project naar het controleren van aanwezige criteria, waardoor er minder tijd is voor inhoudelijke feedback. Daarnaast worden de projecten beoordeeld door meerdere lectoren, waardoor consistente evaluaties moeilijker worden.

\vspace{1em}

Dit onderzoek richt zich expliciet op de lectoren Front-End Web Development en Web Services aan HOGENT als primaire doelgroep. Deze bachelorproef zal hun concrete inzichten bieden in de bruikbaarheid en beperkingen van LLM-ondersteunde code-evaluatie voor deze specifieke casus.

\section{\IfLanguageName{dutch}{Onderzoeksvraag}{Research question}}%
\label{sec:onderzoeksvraag}

Uit bovenstaande probleemstelling komt volgende centrale onderzoeksvraag naar boven:

 \begin{quote}
    \textit{``In welke mate kunnen Large Language Models bruikbare en betrouwbare formatieve feedback genereren op code-kwaliteit van studentenprojecten binnen webontwikkeling en hoe verhouden deze beoordelingen zich tot de evaluaties van ervaren lectoren?''}
\end{quote}

Deze onderzoeksvraag kan opgesplitst worden in meerdere deelvragen, zowel met betrekking tot het probleemdomein, als het oplossingsdomein.

\textbf{\newline Deelvragen probleemdomein:}
\begin{enumerate}
    \item Welke specifieke evaluatiecriteria worden momenteel gebruikt bij de evaluatie van projecten binnen Front-end Web Development en Web Services?
    \item Hoeveel tijd wordt gemiddeld gespendeerd aan het beoordelen van 1 project?
    \item Van de tijd die gespendeerd wordt aan het beoordelen van één project, welk deel is gericht op het controleren van de criteria, en welk deel is gericht op het geven van inhoudelijke feedback?
    \item In welke mate ervaren de lectoren inconsistenties in hun eigen beoordelingen ten opzichte van hun collega's?
\end{enumerate}

\vspace{1em}

\textbf{Deelvragen oplossingsdomein:}
\begin{enumerate}
    \item In welke mate zijn verschillende LLM’s in staat om vooraf gedefinieerde evaluatiecriteria correct te detecteren in de codebase van een studentenproject?
    \item In welke mate komt de door LLM’s gegenereerde feedback overeen met de feedback van lectoren?
    \item Hoe consistent is de feedback van verschillende LLM's bij hetzelfde project?
    \item Op welke criteria scoren LLM’s sterk en op welke scoren ze zwakker bij het evalueren van studentenprojecten?
    \item Wat is de potentiële tijdswinst bij het gebruik van een LLM, ten opzichte van alle projecten handmatig evalueren?
    \item Welke prompt-engineeringtechnieken verhogen de kwaliteit en betrouwbaarheid van LLM-gebaseerde code-evaluatie?
\end{enumerate}

\section{\IfLanguageName{dutch}{Onderzoeksdoelstelling}{Research objective}}%
\label{sec:onderzoeksdoelstelling}

Het beoogde resultaat van dit onderzoek bevat meerdere zaken. Enerzijds een grondig onderbouwd onderzoeksverslag, anderzijds een goed gedocumenteerde Proof-of-Concept(PoC) testomgeving voor het evalueren van studentenprojecten.

\vspace{1em}

Ten eerste moet er een onderzoeksverslag worden opgeleverd, waarin er op een gestructureerde en onderbouwde manier geanalyseerd wordt in welke mate Large Language Models (LLM's) bruikbare en betrouwbare formatieve feedback kunnen genereren op de codekwaliteit van studentenprojecten binnen Front-End Web Development en Web Services. Dit onderzoeksverslag bevat:
\begin{itemize}
    \item een overzicht van de huidige evaluatiecriteria en methodes die door lectoren worden gebruikt
    \item een analyse van verschillende LLM's op vlak van criteriadetectie, consistentie,\ldots
    \item een vergelijking tussen de feedback van ervaren lectoren en de LLM-gegenereerde feedback
    \item de situering van dit onderzoek binnen de AI act, en hoe hier mee moet omgegaan worden als lectoren
    \item conclusies en concrete aanbevelingen voor lectoren, omtrent het inzetten van LLM's bij de evaluatie van studentenprojecten.
\end{itemize}

\vspace{1em}

Ten tweede zal er een functionele PoC opgeleverd worden binnen een goed gedocumenteerde en reproduceerbare testomgeving. Deze PoC zal toestaan om:

\begin{itemize}
    \item vooraf gedefinieerde criteria te gaan detecteren binnen studentenprojecten 
    \item gestructureerde feedback te genereren per criterium
    \item evaluaties te herhalen, zodat consistentie bekeken kan worden
    \item makkelijk te kunnen wisselen van LLM, zodat deze met elkaar kunnen worden vergeleken
\end{itemize} 

\vspace{1em} 

Deze bachelorproef zal dus geen kant en klare evaluatietool opleveren, maar wel een goed onderbouwde PoC en vergelijkende studie. Op basis hiervan kunnen lectoren beslissen in welke mate zij LLM's willen gaan inzetten ter ondersteuning van hun projectevaluaties.

\section{\IfLanguageName{dutch}{Opzet van deze bachelorproef}{Structure of this bachelor thesis}}%
\label{sec:opzet-bachelorproef}

De rest van deze bachelorproef is als volgt opgebouwd:

In Hoofdstuk~\ref{ch:stand-van-zaken} wordt een overzicht gegeven van de stand van zaken binnen het onderzoeksdomein, op basis van een literatuurstudie.

In Hoofdstuk~\ref{ch:methodologie} wordt de methodologie toegelicht en worden de gebruikte onderzoekstechnieken besproken om een antwoord te kunnen formuleren op de onderzoeksvragen.

% TODO: Vul hier aan voor je eigen hoofstukken, één of twee zinnen per hoofdstuk

In Hoofdstuk~\ref{ch:conclusie}, tenslotte, wordt de conclusie gegeven en een antwoord geformuleerd op de onderzoeksvragen. Daarbij wordt ook een aanzet gegeven voor toekomstig onderzoek binnen dit domein.