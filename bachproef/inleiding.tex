%%=============================================================================
%% Inleiding
%%=============================================================================

\chapter{\IfLanguageName{dutch}{Inleiding}{Introduction}}%
\label{ch:inleiding}

De inleiding moet de lezer net genoeg informatie verschaffen om het onderwerp te begrijpen en in te zien waarom de onderzoeksvraag de moeite waard is om te onderzoeken. In de inleiding ga je literatuurverwijzingen beperken, zodat de tekst vlot leesbaar blijft. Je kan de inleiding verder onderverdelen in secties als dit de tekst verduidelijkt. Zaken die aan bod kunnen komen in de inleiding:

\begin{itemize}
  \item context, achtergrond
  \item afbakenen van het onderwerp
  \item verantwoording van het onderwerp, methodologie
  \item probleemstelling
  \item onderzoeksdoelstelling
  \item onderzoeksvraag
  \item \ldots
\end{itemize}

\section{\IfLanguageName{dutch}{Probleemstelling}{Problem Statement}}%
\label{sec:probleemstelling}

Uit je probleemstelling moet duidelijk zijn dat je onderzoek een meerwaarde heeft voor een concrete doelgroep. De doelgroep moet goed gedefinieerd en afgelijnd zijn. Doelgroepen als ``bedrijven,'' ``KMO's'', systeembeheerders, enz.~zijn nog te vaag. Als je een lijstje kan maken van de personen/organisaties die een meerwaarde zullen vinden in deze bachelorproef (dit is eigenlijk je steekproefkader), dan is dat een indicatie dat de doelgroep goed gedefinieerd is. Dit kan een enkel bedrijf zijn of zelfs één persoon (je co-promotor/opdrachtgever).

\section{\IfLanguageName{dutch}{Onderzoeksvraag}{Research question}}%
\label{sec:onderzoeksvraag}

Uit bovenstaande probleemstelling komt volgende centrale onderzoeksvraag naar boven:

 \begin{quote}
    \textit{``In welke mate kunnen Large Language Models bruikbare en betrouwbare formatieve feedback genereren op code-kwaliteit van studentenprojecten binnen webontwikkeling en hoe verhouden deze beoordelingen zich tot de evaluaties van ervaren lectoren?''}
\end{quote}

Deze onderzoeksvraag kan opgesplitst worden in meerdere deelvragen, zowel met betrekking tot het probleemdomein, als het oplossingsdomein.

\textbf{\newline Deelvragen probleemdomein:}
\begin{enumerate}
    \item Welke specifieke evaluatiecriteria worden momenteel gebruikt bij de evaluatie van projecten binnen Front-end Web Development en Web Services?
    \item Hoeveel tijd wordt gemiddeld gespendeerd aan het beoordelen van 1 project?
    \item Van de tijd die gespendeerd wordt aan het beoordelen van één project, welk deel is gericht op het controleren van de criteria, en welk deel is gericht op het geven van inhoudelijke feedback?
    \item In welke mate ervaren de lectoren inconsistenties in hun eigen beoordelingen ten opzichte van hun collega's?
\end{enumerate}

\vspace{1em}

\textbf{Deelvragen oplossingsdomein:}
\begin{enumerate}
    \item In welke mate zijn verschillende LLM’s in staat om vooraf gedefinieerde evaluatiecriteria correct te detecteren in de codebase van een studentenproject?
    \item In welke mate komt de door LLM’s gegenereerde feedback overeen met de feedback van lectoren?
    \item Hoe consistent is de feedback van verschillende LLM's bij hetzelfde project?
    \item Op welke criteria scoren LLM’s sterk en op welke scoren ze zwakker bij het evalueren van studentenprojecten?
    \item Wat is de potentiële tijdswinst bij het gebruik van een LLM, ten opzichte van alle projecten handmatig evalueren?
    \item Welke prompt-engineeringtechnieken verhogen de kwaliteit en betrouwbaarheid van LLM-gebaseerde code-evaluatie?
\end{enumerate}

\section{\IfLanguageName{dutch}{Onderzoeksdoelstelling}{Research objective}}%
\label{sec:onderzoeksdoelstelling}

Wat is het beoogde resultaat van je bachelorproef? Wat zijn de criteria voor succes? Beschrijf die zo concreet mogelijk. Gaat het bv.\ om een proof-of-concept, een prototype, een verslag met aanbevelingen, een vergelijkende studie, enz.

\section{\IfLanguageName{dutch}{Opzet van deze bachelorproef}{Structure of this bachelor thesis}}%
\label{sec:opzet-bachelorproef}

De rest van deze bachelorproef is als volgt opgebouwd:

In Hoofdstuk~\ref{ch:stand-van-zaken} wordt een overzicht gegeven van de stand van zaken binnen het onderzoeksdomein, op basis van een literatuurstudie.

In Hoofdstuk~\ref{ch:methodologie} wordt de methodologie toegelicht en worden de gebruikte onderzoekstechnieken besproken om een antwoord te kunnen formuleren op de onderzoeksvragen.

% TODO: Vul hier aan voor je eigen hoofstukken, één of twee zinnen per hoofdstuk

In Hoofdstuk~\ref{ch:conclusie}, tenslotte, wordt de conclusie gegeven en een antwoord geformuleerd op de onderzoeksvragen. Daarbij wordt ook een aanzet gegeven voor toekomstig onderzoek binnen dit domein.