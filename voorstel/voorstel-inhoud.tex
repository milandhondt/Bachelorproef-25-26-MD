%---------- Inleiding ---------------------------------------------------------

\section{Inleiding}%
\label{sec:inleiding}

% Waarover zal je bachelorproef gaan? Introduceer het thema en zorg dat volgende zaken zeker duidelijk aanwezig zijn:

% \begin{itemize}
  % \item kaderen thema
  % \item de doelgroep
  % \item de probleemstelling en (centrale) onderzoeksvraag
  % \item de onderzoeksdoelstelling
% \end{itemize}

% Denk er aan: een typische bachelorproef is \textit{toegepast onderzoek}, wat betekent dat je start vanuit een concrete probleemsituatie in bedrijfscontext, een \textbf{casus}. Het is belangrijk om je onderwerp goed af te bakenen: je gaat voor die \textit{ene specifieke probleemsituatie} op zoek naar een goede oplossing, op basis van de huidige kennis in het vakgebied.

% De doelgroep moet ook concreet en duidelijk zijn, dus geen algemene of vaag gedefinieerde groepen zoals \emph{bedrijven}, \emph{developers}, \emph{Vlamingen}, enz. Je richt je in elk geval op it-professionals, een bachelorproef is geen populariserende tekst. Eén specifiek bedrijf (die te maken hebben met een concrete probleemsituatie) is dus beter dan \emph{bedrijven} in het algemeen.

% Formuleer duidelijk de onderzoeksvraag! De begeleiders lezen nog steeds te veel voorstellen waarin we geen onderzoeksvraag terugvinden.

% Schrijf ook iets over de doelstelling. Wat zie je als het concrete eindresultaat van je onderzoek, naast de uitgeschreven scriptie? Is het een proof-of-concept, een rapport met aanbevelingen, \ldots Met welk eindresultaat kan je je bachelorproef als een succes beschouwen?

%---------- Stand van zaken ---------------------------------------------------

\section{Literatuurstudie}%
\label{sec:literatuurstudie}

% Hier beschrijf je de \emph{state-of-the-art} rondom je gekozen onderzoeksdomein, d.w.z.\ een inleidende, doorlopende tekst over het onderzoeksdomein van je bachelorproef. Je steunt daarbij heel sterk op de professionele \emph{vakliteratuur}, en niet zozeer op populariserende teksten voor een breed publiek. Wat is de huidige stand van zaken in dit domein, en wat zijn nog eventuele open vragen (die misschien de aanleiding waren tot je onderzoeksvraag!)?

% Je mag de titel van deze sectie ook aanpassen (literatuurstudie, stand van zaken, enz.). Zijn er al gelijkaardige onderzoeken gevoerd? Wat concluderen ze? Wat is het verschil met jouw onderzoek?

% Verwijs bij elke introductie van een term of bewering over het domein naar de vakliteratuur, bijvoorbeeld~\autocite{Hykes2013}! Denk zeker goed na welke werken je refereert en waarom.

% Draag zorg voor correcte literatuurverwijzingen! Een bronvermelding hoort thuis \emph{binnen} de zin waar je je op die bron baseert, dus niet er buiten! Maak meteen een verwijzing als je gebruik maakt van een bron. Doe dit dus \emph{niet} aan het einde van een lange paragraaf. Baseer nooit teveel aansluitende tekst op eenzelfde bron.

% Als je informatie over bronnen verzamelt in JabRef, zorg er dan voor dat alle nodige info aanwezig is om de bron terug te vinden (zoals uitvoerig besproken in de lessen Research Methods).

% Voor literatuurverwijzingen zijn er twee belangrijke commando's:
% \autocite{KEY} => (Auteur, jaartal) Gebruik dit als de naam van de auteur
%   geen onderdeel is van de zin.
% \textcite{KEY} => Auteur (jaartal)  Gebruik dit als de auteursnaam wel een
%   functie heeft in de zin (bv. ``Uit onderzoek door Doll & Hill (1954) bleek
%   ...'')

% Je mag deze sectie nog verder onderverdelen in subsecties als dit de structuur van de tekst kan verduidelijken.

%---------- Methodologie ------------------------------------------------------
\section{Methodologie}%
\label{sec:methodologie}

%TODO: Aanvullen met tijdsbestedingen per fase
%TODO: Visuele voorstelling van tijdsbesteding per fase

%===================
%    INLEIDING
%===================
Deze bachelorproef zal uitgewerkt worden in verschillende fases. In elk van deze fases wordt een andere techniek gebruikt, en wordt de focus op een ander onderdeel van het probleem gelegd. Binnen de methodologie wordt bij de tijdsinschatting gesproken over werkdagen. Dit is de dag in de week dat er verplicht aan de bachelorproef gewerkt moet worden. Als een fase dus 2 werkdagen in beslag neemt, dan zijn dit eigenlijk 2 weken, waarbij dus in elke week 1 dag aan de bachelorproef wordt gewerkt. In totaal zullen er 14 werkdagen zijn.

%===================
%      FASE 1
%===================
\paragraph{Fase 1: Probleemdomein onderzoeken (1 werkdag)}
In de eerste fase van dit onderzoek zal het probleemdomein verder worden ontdekt. Hiervoor zal er een interview worden afgenomen met een lector Web Services & Front-end web development. Dit is namelijk de primaire doelgroep van deze bachelorproef. Dit interview wordt gedaan voor 3 redenen. De eerste reden is het verder verdiepen in het probleem en de evaluatiecriteria en methoden die de lectoren vandaag hanteren. De tweede reden is om de nood aan onderzoek, en de relevantie van het onderzoek verder te verduidelijken. Als laatste zou er bij dit interview gepolst kunnen worden of er studentenprojecten van verschillende kwaliteiten en beoordelingen van verschillende lectoren kunnen opgeleverd worden, om in een volgende fase het onderzoek verder te kunnen uitwerken.

%===================
%      FASE 2
%===================
\paragraph{Fase 2: Literatuurstudie (2 werkdagen)}
De tweede fase bevat de literatuurstudie. Hierin zal er gekeken worden naar welke LLMs (Large Languague Models) er het meest geschikt zijn voor het analyseren van code kwaliteit, en het geven van consistente en duidelijke feedback. Deze literatuurstudie zal zowel lokale als cloud modellen onderzoeken. Hieruit zal een longlist worden opgesteld met de modellen die vergeleken zullen worden in een volgende onderzoeksfase. Hier zullen zowel voor- als nadelen worden opgelijst, alsook technische details zoals token en context limitaties, wat zeer belangrijk zal zijn voor de uiteindelijke gebruiksvriendelijkheid van het model. Als laatste zal er ook gekeken worden naar hoe de te grote projecten zullen worden opgesplitst, hier bestaan er namelijk ook verschillende manieren voor.

%===================
%      FASE 3
%===================
\paragraph{Fase 3: Opzetten testomgeving (4 werkdagen)}
Tijdens de derde fase zal de testomgeving worden opgezet. Dit omvat het opzetten van een script dat de studentenprojecten opsplitst om te voldoen aan de token limieten van het model en het opzetten van een script dat de gesplitste delen automatisch verstuurt naar verschillende modellen met een identieke prompt. Dit alles moet goed gedocumenteerd worden, zodat het onderzoek reproduceerbaar en herbruikbaar blijft.

%===================
%      FASE 4
%===================
\paragraph{Fase 4: Kwantitatieve evaluatie van LLMs (5 werkdagen)} 
De vierde fase zal het individueel evalueren van verschillende LLMs omvatten. Op basis van de literatuurstudie zullen de meest relevante en bruikbare modellen worden gekozen, en deze zullen individueel worden beoordeeld. Hierbij zal gekeken worden naar het detecteren van evaluatiecriteria, of de output klopt, of de output consistent is met die van leerkrachten, en of de output consistent is wanneer hetzelfde project meerdere keren geëvalueerd wordt. Als laatste zal er gekeken worden naar hoeveel potentiële tijdswinst verbonden zit aan LLM ondersteunde code-evaluatie. Wanneer alle modellen individueel zijn beoordeeld, dan volgt de vergelijkende studie. Hierin worden de verschillende modellen met elkaar vergeleken op alle bovenstaande criteria.

%===================
%      FASE 5
%===================
\paragraph{Fase 5: }
QSDF

- Ollama gebruiken voor het gemak, verschillende modellen kunnen makkelijk geswitcht worden en exact dezelfde prompt gebruiken.

- Ollama heeft ook het voordeel dat je hier specifiek llm's kunt draaien die puur geschikt zijn voor code doeleinden, dus dit zou betere output kunnen geven dan generieke llm's.

- Misschien VIC inschakelen om krachtigere modellen te kunnen runnen?

% Hier beschrijf je hoe je van plan bent het onderzoek te voeren. Welke onderzoekstechniek ga je toepassen om elk van je onderzoeksvragen te beantwoorden? Gebruik je hiervoor literatuurstudie, interviews met belanghebbenden (bv.~voor requirements-analyse), experimenten, simulaties, vergelijkende studie, risico-analyse, PoC, \ldots?

% Valt je onderwerp onder één van de typische soorten bachelorproeven die besproken zijn in de lessen Research Methods (bv.\ vergelijkende studie of risico-analyse)? Zorg er dan ook voor dat we duidelijk de verschillende stappen terug vinden die we verwachten in dit soort onderzoek!

% Uit dit onderdeel moet duidelijk naar voor komen dat je bachelorproef ook technisch voldoen\-de diepgang zal bevatten. Het zou niet kloppen als een bachelorproef informatica ook door bv.\ een student marketing zou kunnen uitgevoerd worden.

% Je beschrijft ook al welke tools (hardware, software, diensten, \ldots) je denkt hiervoor te gebruiken.

% Probeer ook een tijdschatting te maken. Hoe lang zal je met elke fase van je onderzoek bezig zijn en wat zijn de concrete \emph{deliverables} in elke fase?

\section{Verwacht resultaat, conclusie}%
\label{sec:verwachte_resultaten}

% Hier beschrijf je welke resultaten je verwacht. Als je metingen en simulaties uitvoert, kan je hier al mock-ups maken van de grafieken samen met de verwachte conclusies. Benoem zeker al je assen en de onderdelen van de grafiek die je gaat gebruiken. Dit zorgt ervoor dat je concreet weet welk soort data je moet verzamelen en hoe je die moet meten.

% Wat heeft de doelgroep van je onderzoek aan het resultaat? Op welke manier zorgt jouw bachelorproef voor een meerwaarde?

% Hier beschrijf je wat je verwacht uit je onderzoek, met de motivatie waarom. Het is \textbf{niet} erg indien uit je onderzoek andere resultaten en conclusies vloeien dan dat je hier beschrijft: het is dan juist interessant om te onderzoeken waarom jouw hypothesen niet overeenkomen met de resultaten.

