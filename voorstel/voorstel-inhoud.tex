%---------- Inleiding ---------------------------------------------------------

\section{Inleiding}%
\label{sec:inleiding}

De afgelopen jaren hebben Large Language Models (LLMs) zoals ChatGPT, Gemini, Claude, Deepseek en GitHub Copilot een enorme impact gecreëerd op bijna alles en iedereen. AI valt bij velen niet meer weg te denken uit het dagelijks leven, en wordt elke dag opnieuw door ontelbare aantallen mensen ingezet om hun taken te vergemakkelijken. Zo ook door studenten Full Stack Development binnen de opleiding Toegepaste Informatica aan HOGENT. Deze studenten zetten LLMs steeds vaker in om code te begrijpen, debuggen, zelfs schrijven. 

\newline Maar hoe goed zijn die LLMs in het begrijpen van code? Hoe goed is de feedback die ze genereren? Is deze wel correct? Is deze bruikbaar? Dat zijn allemaal vragen waar de studenten zelf weinig bij stil staan, maar dit is wel relevant. Binnen de opleidingsonderdelen Front-End Web Development en Web Services worden er namelijks per semester tientallen projecten ingediend, en deze worden momenteel nog volledig handmatig beoordeeld door de lectoren. Die manuele beoordeling is enorm tijdsintensief, en repetitief. Lectoren moeten dezelfde criteria, denk bijvoorbeeld aan structuur, consistentie, documentatie, foutafhandeling, testing,\ldots telkens opnieuw toepassen op elk project. Hierdoor gaat eigenlijk een groot deel van hun tijd naar het puur controleren van de aanwezigheid van bepaalde onderdelen in het project, en niet naar het geven van inhoudelijke feedback. Verder is het ook lastig voor de lectoren om consistent feedback te geven. Zo hangt de feedback die een bepaald project zou krijgen ook nog eens af van welke persoon het beoordeeld.

\newline Deze concrete probleemsituatie geeft aanleiding tot de vraag of lectoren LLMs niet zouden kunnen gebruiken ter ondersteuning, door feedback te genereren over de code-kwaliteit van een studentenproject. Hoe bruikbaar, betrouwbaar, consistent of accuraat deze feedback is, en hoe consistent ze is in vergelijking met de feedback van een lector is echter nog onduidelijk.

\newline Daaruit vloeit dus ook volgende onderzoeksvraag voort:

``In welke mate kunnen Large Language Models bruikbare en betrouwbare formatieve feedback genereren op code-kwaliteit van studentenprojecten binnen webontwikkeling, en hoe verhouden deze beoordelingen zich tot de evaluaties van ervaren lectoren?''

% Waarover zal je bachelorproef gaan? Introduceer het thema en zorg dat volgende zaken zeker duidelijk aanwezig zijn:

% \begin{itemize}
  % \item kaderen thema
  % \item de doelgroep
  % \item de probleemstelling en (centrale) onderzoeksvraag
  % \item de onderzoeksdoelstelling
% \end{itemize}

% Denk er aan: een typische bachelorproef is \textit{toegepast onderzoek}, wat betekent dat je start vanuit een concrete probleemsituatie in bedrijfscontext, een \textbf{casus}. Het is belangrijk om je onderwerp goed af te bakenen: je gaat voor die \textit{ene specifieke probleemsituatie} op zoek naar een goede oplossing, op basis van de huidige kennis in het vakgebied.

% De doelgroep moet ook concreet en duidelijk zijn, dus geen algemene of vaag gedefinieerde groepen zoals \emph{bedrijven}, \emph{developers}, \emph{Vlamingen}, enz. Je richt je in elk geval op it-professionals, een bachelorproef is geen populariserende tekst. Eén specifiek bedrijf (die te maken hebben met een concrete probleemsituatie) is dus beter dan \emph{bedrijven} in het algemeen.

% Formuleer duidelijk de onderzoeksvraag! De begeleiders lezen nog steeds te veel voorstellen waarin we geen onderzoeksvraag terugvinden.

% Schrijf ook iets over de doelstelling. Wat zie je als het concrete eindresultaat van je onderzoek, naast de uitgeschreven scriptie? Is het een proof-of-concept, een rapport met aanbevelingen, \ldots Met welk eindresultaat kan je je bachelorproef als een succes beschouwen?

\section{Literatuurstudie}%
\label{sec:literatuurstudie}

% Hier beschrijf je de \emph{state-of-the-art} rondom je gekozen onderzoeksdomein, d.w.z.\ een inleidende, doorlopende tekst over het onderzoeksdomein van je bachelorproef. Je steunt daarbij heel sterk op de professionele \emph{vakliteratuur}, en niet zozeer op populariserende teksten voor een breed publiek. Wat is de huidige stand van zaken in dit domein, en wat zijn nog eventuele open vragen (die misschien de aanleiding waren tot je onderzoeksvraag!)?

% Je mag de titel van deze sectie ook aanpassen (literatuurstudie, stand van zaken, enz.). Zijn er al gelijkaardige onderzoeken gevoerd? Wat concluderen ze? Wat is het verschil met jouw onderzoek?

% Verwijs bij elke introductie van een term of bewering over het domein naar de vakliteratuur, bijvoorbeeld~\autocite{Hykes2013}! Denk zeker goed na welke werken je refereert en waarom.

% Draag zorg voor correcte literatuurverwijzingen! Een bronvermelding hoort thuis \emph{binnen} de zin waar je je op die bron baseert, dus niet er buiten! Maak meteen een verwijzing als je gebruik maakt van een bron. Doe dit dus \emph{niet} aan het einde van een lange paragraaf. Baseer nooit teveel aansluitende tekst op eenzelfde bron.

% Als je informatie over bronnen verzamelt in JabRef, zorg er dan voor dat alle nodige info aanwezig is om de bron terug te vinden (zoals uitvoerig besproken in de lessen Research Methods).

% Voor literatuurverwijzingen zijn er twee belangrijke commando's:
% \autocite{KEY} => (Auteur, jaartal) Gebruik dit als de naam van de auteur
%   geen onderdeel is van de zin.
% \textcite{KEY} => Auteur (jaartal)  Gebruik dit als de auteursnaam wel een
%   functie heeft in de zin (bv. ``Uit onderzoek door Doll & Hill (1954) bleek
%   ...'')

% Je mag deze sectie nog verder onderverdelen in subsecties als dit de structuur van de tekst kan verduidelijken.

\section{Methodologie}%
\label{sec:methodologie}

%TODO: Visuele voorstelling van tijdsbesteding per fase

%===================
%    INLEIDING
%===================
Deze bachelorproef zal uitgewerkt worden in verschillende fases. In elk van deze fases wordt een andere techniek gebruikt, en wordt de focus op een ander onderdeel van het probleem gelegd. In totaal zijn er 14 werkdagen voorzien voor de bachelorproef. Aangezien er 1 werkdag per week besteed wordt aan het onderzoek, komt 1 werkdag overeen met 1 week.

Een overzicht van de fasen wordt weergegeven in Figuur~\ref{fig:gantt}.\newline

%===================
%      FASE 1
%===================
\paragraph{Fase 1: Probleemdomein onderzoeken (1 werkdag)}
In de eerste fase van dit onderzoek zal het probleemdomein verder worden ontdekt. Hiervoor zal er een interview worden afgenomen met een lector Web Services & Front-end web development. Dit is namelijk de primaire doelgroep van deze bachelorproef. Dit interview wordt gedaan voor 3 redenen: 
\begin{itemize}
    \item Het verder verdiepen in het probleem en de evaluatiecriteria en methoden die de lectoren vandaag hanteren.
    \item De nood aan onderzoek, en de relevantie van het onderzoek verder te verduidelijken.
    \item Nagaan of er studentenprojecten van verschillende kwaliteitsniveaus kunnen worden opgeleverd. Dit zou bij voorkeur ook door verschillende lectoren gedaan worden, om de consistentie van de evaluaties in acht te kunnen nemen.
\end{itemize}

Het resultaat van deze fase is een duidelijker inzicht binnen het probleemdomein, een lijst met de verschillende evaluatiecriteria, alsook verschillende studentenprojecten die gebruikt kunnen worden als testprojecten in een volgende fase.

%===================
%      FASE 2
%===================
\paragraph{Fase 2: Literatuurstudie (2 werkdagen)}
De tweede fase bevat de literatuurstudie. Hierin zal er gekeken worden naar welke LLMs (Large Languague Models) er het meest geschikt zijn voor het analyseren van code-kwaliteit, en het geven van consistente en duidelijke feedback. Deze literatuurstudie zal zowel lokale als cloud modellen onderzoeken, via Ollama. Hieruit zal een longlist worden opgesteld met de potentiële modellen die in aanmerking komen voor verder onderzoek. Bij het kiezen van de modellen wordt er aandacht besteed aan:
\begin{itemize}
    \item Voordelen
    \item Nadelen
    \item Token- en contextlimitaties
    \item Geschiktheid voor code-analyse
\end{itemize}
Als laatste zal er ook gekeken worden naar hoe de te grote projecten zullen worden opgesplitst, hier bestaan er namelijk ook verschillende manieren voor. Denk hierbij aan per bestand, per laag, via chunking, via een samenvatting,\ldots

\noindent Het resultaat van deze fase is een duidelijke, goed onderbouwde lijst van bruikbare modellen, inclusief hun technische eigenschappen en eventuele beperkingen.

%===================
%      FASE 3
%===================
\paragraph{Fase 3: Opzetten testomgeving (4 werkdagen)}
Tijdens de derde fase zal de testomgeving worden opgezet. Dit omvat het opzetten van een script dat de studentenprojecten opsplitst om te voldoen aan de token limieten van het model, alsook het opzetten van een script dat de gesplitste delen automatisch verstuurt naar verschillende modellen met een identieke prompt. Binnen deze fase zal er ook gekeken worden om eventueel samen werken met het VIC (Virtual IT Company) van HOGENT. Hier is het namelijk misschien mogelijk om sterkere servers te gebruiken om de rekenkracht van een model te verhogen, iets wat voor dit onderzoek wel van pas zou kunnen komen. In deze fase is het zeer belangrijk dat alles goed gedocumenteerd wordt, zodat het onderzoek reproduceerbaar en herbruikbaar blijft.

Het resultaat van deze fase is een volledig werkende proof-of-concept die klaar is om de vergelijkende analyse te gaan uitvoeren. Dit, inclusief de documentatie, zou beschikbaar moeten zijn op een GitHub repository.

%===================
%      FASE 4
%===================
\paragraph{Fase 4: Kwantitatieve evaluatie van LLMs (5 werkdagen)} 
De vierde fase zal het individueel evalueren van verschillende LLMs omvatten. Op basis van de literatuurstudie zullen de meest relevante en bruikbare modellen worden gekozen, en deze zullen individueel worden beoordeeld. Hierbij zal gekeken worden naar 4 punten:
\begin{itemize}
    \item Het detecteren van evaluatiecriteria.
    \item Klopt de output?
    \item Is de output consistent met die van leerkrachten?
    \item Is de output consistent wanneer hetzelfde project meerdere keren geëvalueerd wordt?
    \item Hoeveel potentiële tijdswinst zit er verbonden aan LLM ondersteunde code-evaluatie?
\end{itemize}

Wanneer alle modellen individueel zijn beoordeeld, dan volgt de vergelijkende studie. Hierin worden de verschillende modellen met elkaar vergeleken op alle bovenstaande criteria.

Het resultaat van deze fase is een uitgebreide analyse en een grondige vergelijking op basis van de bekomen resultaten.

%===================
%      FASE 5
%===================
\paragraph{Fase 5: Conclusie (2 werkdagen)}
In de laatste fase zal de conclusie worden opgebouwd. De resultaten van al het onderzoek uit voorgaande fases worden gebundeld om objectief aan te tonen of LLMs mogelijks gebruikt kunnen worden voor het evalueren van studentenprojecten binnen het afgebakende probleemdomein. Daarnaast worden de verschillende voor- en nadelen van de geëvalueerde modellen toegelicht, om de keuze voor lectoren te vergemakkelijken.

Het resultaat van deze fase is een volledig uitgewerkte conclusie, aangevuld met de bevindingen, aanbevelingen voor lectoren en eventuele suggesties voor verder onderzoek.

\section{Verwacht resultaat, conclusie}%
\label{sec:verwachte_resultaten}

% Hier beschrijf je welke resultaten je verwacht. Als je metingen en simulaties uitvoert, kan je hier al mock-ups maken van de grafieken samen met de verwachte conclusies. Benoem zeker al je assen en de onderdelen van de grafiek die je gaat gebruiken. Dit zorgt ervoor dat je concreet weet welk soort data je moet verzamelen en hoe je die moet meten.

% Wat heeft de doelgroep van je onderzoek aan het resultaat? Op welke manier zorgt jouw bachelorproef voor een meerwaarde?

% Hier beschrijf je wat je verwacht uit je onderzoek, met de motivatie waarom. Het is \textbf{niet} erg indien uit je onderzoek andere resultaten en conclusies vloeien dan dat je hier beschrijft: het is dan juist interessant om te onderzoeken waarom jouw hypothesen niet overeenkomen met de resultaten.

